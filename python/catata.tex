if and else
percabngan jika da maka

inputan

mengecek tipe data menggunkan type

TUGAS
1.Cupy_position  cara agar tidak manual menggunkana Libbary
2. apakah kamu yakin dengan jawaban tersebut [y/n]

HARI KE 2

PENGENALAN Typecasting, list, Libbarys

1. Typecasting adlaah merubah sebuah nilai pada variable
2. List atau di sebut array adalah sekumpulan pada variable
3. libbary adalah sebuah tools untuk mempermudah untuk coding

exit adalah fungsi untuk menghentikan program
len untuk mengetahui panjang pada data
selow copy &copy adaah copy data pada sebelum nya ini lumayan di butuhkan

TUGAS :
- tidak Memuculkan Cuypuy agar user bisa menebak dan tidak bisa menang terus
- Membuang [] dan koma pada goa nya kata kunci join.array
  ini code nya
  membuat varaibel = " ".join(nama_variable)
  error yang lama adalah karena
  - TypeError: 'str' object does not support item assignment : str tidak bisa menangangi penugasasn error
  Cara Perbaiki nya adalah
  1. menggunakan split : untuk mememcahkan sebuah list / array berdasarkan pemerintah tertentu 
  2. pemanggilan variable 

  Catatan :
  1. Harus Lebih teliti lagi saat ngoding dan juga awas di variable soal kamu gampang lupa 
  2. buat logika yang sederhana
  3. berusaha lah untuk baca dokumntasi dan kurangi AI , Ai Boleh tapi jangan terlalu extrim
Membaut Cuypuy masuk ke visual gambar nya yaitu |_| lubang goa nya

1. Keluarkan Visual gambar nya dulu
  - 
